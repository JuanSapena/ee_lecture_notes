\chapter{Why this book?}
\clabel{Why}

%\section{Summary}
%\seclabel{Summary}
{\it In this introductory chapter we lay the conceptual foundation for the rest 
of what we have to say about redeveloping economic theory. In \secref{The_game} we play a simple coin-toss game and analyze it 
numerically, by Monte Carlo simulation, and analytically, with pen and paper. The game motivates the 
introduction of the expectation value and the time average, which in turn 
lead to a discussion of ergodic properties. The ergodicity question  -- whether time averages are identical to expectation values -- will turn out to be the key to our redevelopment of formal economics. This is because ergodicity hadn't been established as a concept when the original formalism was developed. We note the importance of rates as ergodic observables.
This section also introduces the concepts of a random variable, a stochastic process, scalars as representations of transitive preferences, logarithms and exponentials, and dimensional analysis.

In \secref{Brownian_motion} we notice that wealth on logarithmic scales follows a random walk in our game, and we relate this to Brownian motion, as the continuous-time limit of the random walk. This allows us to introduce Brownian motion and its scaling properties that are robust enough to yield insights into more complicated models.

Finally, we ask in \secref{Geometric_Brownian} what wealth in our game is doing in the continuum limit but on linear scales. This takes us to geometric Brownian motion, which will be our starting point for much of the rest of these lectures. We derive ensemble-average and time-average growth rates for geometric Brownian motion, by explicitly taking the continuous-time limit, and then state the key result of \Ito calculus, \eref{Ito_process} and \eref{Ito}, which allows an easier derivation of these growth rates and will be relied on in later chapters.

Some historical perspective is provided to understand the prevalence or
absence of key concepts in modern economic theory and other fields.
The emphasis is on introducing key concepts and useful machinery, with more formal treatments and applications
in later chapters.}
\newpage

\section{The game}
\seclabel{The_game}
Imagine we offer you the following game: we toss a coin, and if it comes 
up heads we increase your monetary wealth by 50\%; if 
it comes up tails we reduce your wealth by 40\%. We're not only 
doing this once, we will do it many times, for example 
once per week for the rest of your life. Would you accept 
the rules of our game? Would you submit your wealth to 
the dynamic our game will impose on it?

Your answer to this question is up to you and will be 
influenced by many factors, such as the importance 
you attach to wealth that can be measured in monetary 
terms, whether you like the thrill of gambling, your 
religion and moral convictions and so on.

In these notes we will mostly ignore these factors.
We will build an extremely simple model of your 
wealth, which will lead to an extremely simple and 
powerful model of the way you make decisions that affect 
your wealth. We are interested in analyzing the 
game mathematically, which requires a translation 
of the game into mathematics. We choose the 
following translation: we introduce the key 
variable, $\x(\t)$, which we refer to as ``wealth''. 
We refer to $\t$ as ``time''. It should be kept in mind that
``wealth'' and ``time'' are just names that we've given to 
mathematical objects. We have chosen these names because
we want to compare the behaviour of the mathematical
objects to the behaviour of wealth over time, but
we emphasize that we're building a model -- whether we write $\x(\t)$, 
or $\text{wealth}(\text{time})$, or \smiley(\lightning) makes no difference to the mathematics. 

The usefulness of our model will be different in different circumstances, 
ranging from completely meaningless to very helpful. There is no 
substitute for careful consideration of any given situation, and
labeling mathematical objects in one way or another is certainly none.

Having got these words of warning out of the way, we define our 
model\footnote{For those in the know: ``our'' coin toss is a discrete version of geometric Brownian motion, the workhorse model of financial mathematics and much more.} of the dynamics of your wealth under the rules we just specified. 
At regular intervals of duration $\dt$ we randomly generate a 
factor $\gr(\t)$ with each possible value
$\gr_i\in \{0.6, 1.5\}$ occurring with probability 1/2, 

\be 
\gr(\t) = \begin{cases}
0.6 &\text{with probability  1/2}\\
1.5 &\text{with probability 1/2}
\end{cases}
\elabel{law}
\ee
and multiply current wealth by that factor, so that
\be
\x(\t)=\gr(\t)\x(\t-\d \t).
\elabel{gamble}
\ee

\section*{Summary of \cref{Why}}

In this chapter we have introduced the following key concepts:
\bi
\item[Random variable]
A random variable $Y$ is a set of pairs of possible values and  corresponding probabilities, $Y=\{(y_1, p_1),(y_2, p_2)...\}$.
The sets may be discrete or continuous. We stressed that a random variable is an a-temporal concept. It's just a bunch of possible values and their weights (probabilities). In real life we often this of generating instances of random variables as time passes, but this is not part of the formal definition of a random variable.

\item[Expectation value]
The expectation value of a random variable is the weighted sum $\ave{Y}=\int y \mathcal{P}_Y(y) dy$, where $\mathcal{P}_Y$ has atomic point masses in the discrete case, which means we can express the integral as $\ave{Y}=\sum_i y_i p_i$.

The expectation value is also called the ensemble average, which reflects a physical interpretation: imagine (infinitely) many possible worlds, identical safe for the value taken by the the random variable $Y$. Those values are represented in the superverse of many worlds in proportion to their probabilties. Averaging $y$ over the ensemble of universes then gives the expectation value.
\ei